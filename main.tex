\documentclass{beamer}
\usetheme{metropolis}
\usepackage{graphicx} % Required for inserting images
\usepackage{amsmath} % Required for inserting images
\usepackage{braket}
\usepackage{xcolor}
\usepackage{soul}
\usepackage{quantikz}
\usepackage{amssymb} % For \iff
\usepackage{caption}
\usepackage[backend=bibtex,style=authoryear]{biblatex}

\addbibresource{./resources/references.bib}

\newcommand{\bigfrac}[2]{\genfrac{}{}{}{0}{#1}{#2}}
\newcommand{\hlc}[2]{%
    {\sethlcolor{#1}\hl{#2}}%
}
\newcommand{\X}{\begin{bmatrix}0 & 1 \\ 1 & 0\end{bmatrix}}
\newcommand{\Y}{\begin{bmatrix}0 & -i \\ i & 0\end{bmatrix}}
\newcommand{\Z}{\begin{bmatrix}1 & 0 \\ 0 & -1\end{bmatrix}}
\newcommand{\Har}{
    \genfrac{}{}{}{0}{1}{\sqrt{2}}
    \begin{bmatrix}1 & 1 \\ 1 & -1\end{bmatrix}
}
\newcommand{\ketzero}{\ket{0}}
\newcommand{\ketone}{\ket{1}}
\newcommand{\brazero}{\bra{0}}
\newcommand{\braone}{\bra{1}}
\newcommand{\matzero}{\begin{bmatrix}1 \\ 0\end{bmatrix}}
\newcommand{\matone}{\begin{bmatrix}0 \\ 1\end{bmatrix}}

\usefonttheme{professionalfonts}
% \everymath{\mathnormal}

\title{Trapped Ion Quantum Computing}
\author{David Skocic}
\date{December 9 2025}

\begin{document}

\maketitle
\section{Basics of Quantum Computing}
\begin{frame}{Qubits}
	\begin{itemize}
		\item Analog to the classical bit (\textcolor{red}{Qu}\textcolor{blue}{bit} = \textcolor{red}{Qu}antum \textcolor{blue}{bit})
		\item Any two-state quantum system can theoretically be a qubit
		\item We use our familiar Dirac notation to represent states
	\end{itemize}
	\begin{center}
		$\ketzero = \matzero$
		and
		$\ketone = \matone$
	\end{center}

	\begin{itemize}
		\item Qubits can have a superposition unlike a classical bit
	\end{itemize}
	\begin{center}
		$\begin{bmatrix}\alpha \\ \beta\end{bmatrix} = \alpha\ketzero + \beta\ketone$
	\end{center}
\end{frame}
\begin{frame}{Electron Spin}
	\begin{itemize}
		\item Electron spin is a familiar two-state quantum system
		\item Recall our spin operators for the different dimensions
	\end{itemize}
	\begin{center}
		$S_x = \genfrac{}{}{}{0}{\hbar}{2}\X$
		\hspace{20pt}
		$S_y = \genfrac{}{}{}{0}{\hbar}{2}\Y$
		\hspace{20pt}
		$S_z = \genfrac{}{}{}{0}{\hbar}{2}\Z$
	\end{center}
	\begin{itemize}
		\item $\matzero$ = $\ketzero$ and $\matone$ = $\ketone$ are the z-basis eigenvectors
		\item Observe how these basis vectors can map to the Pauli spin matrices
	\end{itemize}
	\begin{center}
		$\ketzero\brazero + \ketone\braone = I$
		\hspace{20pt}
		$\ketzero\braone + \ketone\brazero = \sigma_x$ \\
		$i(\ketzero\braone - \ketone\brazero) = \sigma_y$
		\hspace{20pt}
		$\ketone\braone - \ketzero\brazero= \sigma_z$
	\end{center}
\end{frame}

\begin{frame}{Unitary Operators}
	\begin{itemize}
		\item The evolution of a quantum system is described by unitary operators
	\end{itemize}
	\begin{center}
		$\ket{\psi_t} = U(t,t_0)\ket{\psi_0}$
	\end{center}
	\begin{itemize}
		\item Unitary Operators are \textit{mathematically reversible}:
	\end{itemize}
	\begin{center}
		$UU^\dagger=I$
	\end{center}
	\begin{itemize}
		\item You can apply unitary operators to a qubit to alter its state in a desirable manner
	\end{itemize}
\end{frame}

\begin{frame}{Quantum Gates}
	\begin{itemize}
		\item In quantum computing, unitary operators are usually called \textbf{quantum gates}
		\item Many of the basic quantum gates are look very familiar!
	\end{itemize}
	\begin{center}
		$X = \X$
		\hspace{20pt}
		$Y = \Y$
		\hspace{20pt}
		$Z = \Z$
	\end{center}
	\begin{itemize}
		\item There are also less familiar gates such as the \textit{Hadamard} Gate:
	\end{itemize}
	\begin{center}
		$H= \Har$
	\end{center}
\end{frame}

\begin{frame}{Quantum Gates}
	\begin{itemize}
		\item The X gate is also called the NOT gate:
	\end{itemize}
	\begin{center}
		$NOT\ketzero=\ketone$ and $NOT\ketone=\ketzero$
		\\[12pt]
		$\X
			\begin{bmatrix}\alpha\\\beta\end{bmatrix} =
			\begin{bmatrix}\beta\\\alpha\end{bmatrix}$
	\end{center}
	\begin{itemize}
		\item The Hadamard gate creates a superposition:
	\end{itemize}
	\begin{center}
		$\Har\matzero =
			\genfrac{}{}{}{0}{1}{\sqrt{2}}
			\begin{bmatrix}1\\1\end{bmatrix}=
			\genfrac{}{}{}{0}{1}{\sqrt{2}}
			(\ketzero+\ketone)$
	\end{center}
\end{frame}
\begin{frame}{Quantum Circuits}
	\begin{itemize}
		\item A series of quantum gates constitutes a \textbf{quantum circuit}
		\item It is standard to represent these as if they were on a wire
	\end{itemize}

	\begin{center}
		\begin{quantikz}
			\ketzero & \gate{H} & \gate{Y} & \gate{Z} & \meter{}
		\end{quantikz}
		\\\rotatebox[origin=c]{90}{{$\iff$}}\\
		$ZYH\ketzero=\Z\Y\Har\matzero=\genfrac{}{}{}{0}{1}{\sqrt{2}}\begin{bmatrix}-i\\-i\end{bmatrix}$
	\end{center}
\end{frame}

\section{Basics of Trapped Ion Quantum Systems}

\begin{frame}{Why Use Trapped Ions?}
	\begin{itemize}
		\item Qubits need to be isolated from the rest of the universe
		\item An ion is much easier to manipulate than a neutral atom
		\item They have exceptionally long coherence times
		\item Quantum gates have been implemented with exceptional fidelity
		\item Quantum states are easy to prepare and measure
		\item The physics behind them is well understood and relatively simple
	\end{itemize}
\end{frame}

\begin{frame}{Encoding a Trapped Ion with a Qubit}
	Take any two long-lived states of different energies $\ket{g}$ and $\ket{e}$\\[4pt]
	Just like spin, these states can map to the Pauli spin matrices:
	\begin{center}
		$\ket{g}\bra{g} + \ket{e}\bra{e} = I$
		\hspace{20pt}
		$\ket{g}\bra{e} + \ket{e}\bra{g} = \sigma_x$ \\
		$i(\ket{g}\bra{e} - \ket{e}\bra{g}) = \sigma_y$
		\hspace{20pt}
		$\ket{e}\bra{e} - \ket{g}\bra{g} = \sigma_z$
	\end{center}
	We can think of $\ket{g}$ as $\ketzero$ and $\ket{e}$ as $\ketone$\\[4pt]
	The two-level Hamiltonian is represented by
	\[
		\hat{H} = \hbar (\omega_g\ket{g}\bra{g}+\omega_e\ket{e}\bra{e}) \rightarrow
		\hat{H} = \hbar\bigfrac{\omega_0}{2}\sigma_z
	\]
\end{frame}

\begin{frame}{Measuring a Qubit}
	\begin{itemize}
		\item Detection of states involves photon scattering\\[4pt]
		\item States are carefully chosen so one "bright" state scatters incoming photons while the other has no effect\\[4pt]
		\item The trajectories of the photons are detected, making is easy to tell which qubits scattered the photons
	\end{itemize}
\end{frame}

\begin{frame}{What Ions Make Good Qubits?}
	\begin{itemize}
		\item There are a few features that are desirable in an ion:
		      \begin{itemize}
			      \item Hydrogen-like atomic structure
			            \vspace{6pt}
			      \item Singly ionized species
			            \vspace{6pt}
			      \item Metastable states in the visible spectrum
		      \end{itemize}
		\item This makes heavier alkali earth metals a common choice
	\end{itemize}
\end{frame}

\begin{frame}{Rabi Frequencies}
	The Rabi frequency comes up when a two-state system is placed in an electric field\\[4pt]
	This frequency is the rate at which the the system 'flops' between states\\[4pt]
	Rabi frequencies are a way to measure the strength of the coupling of the states and the field applied defined by\\[4pt]
	\[\Omega = -\frac{\bra{e}\vec{\mu} \cdot \vec{\epsilon}\hspace{2pt}E_0\ket{g}}{\hbar} \]
\end{frame}

\section{Classical Ion Trapping}

\begin{frame}{Classical Ion Trapping}
	\begin{itemize}
		\item Static electric fields cannot trap an ion
		      \begin{itemize}
			      \item Gauss's law $\nabla \cdot E = 0 \rightarrow$  electric potential has no local minima
			      \item This is overcome by varying the electric field with time
		      \end{itemize}
		\item These traps typically have potentials oscillating in the radio frequency range
		\item Construct a trap with a time-varying and time-independent part:
	\end{itemize}
	\[
		V(x,y,z,t) = U\frac{1}{2}(\alpha x^2 + \beta y^2 + \gamma z^2) +
		\skew{4}\tilde{U}\cos(\omega_{rf}t)\frac{1}{2}(\alpha' x^2 + \beta' y^2 + \gamma' z^2)
	\]
\end{frame}

\begin{frame}{Classical Ion Trapping}
	\begin{center}
		\small
		\[
			V(x,y,z,t) = U\frac{1}{2}(\alpha x^2 + \beta y^2 + \gamma z^2) +
			\skew{4}\tilde{U}\cos(\omega_{rf}t)\frac{1}{2}(\alpha' x^2 + \beta' y^2 + \gamma' z^2)
		\]
	\end{center}
	Laplace's Equation ($\nabla^2 V = 0$) says:
	\begin{center}
		$\alpha+\beta+\gamma= 0$ \\
		$\alpha'+\beta'+\gamma' = 0$
	\end{center}
	Which is underconstrained -- there are many ways to place electrodes
\end{frame}

\begin{frame}{Linear Paul Trap}
	\begin{figure}
		\includegraphics[width=0.45\textwidth]{./resources/Screenshot 2025-12-09 at 4.44.50 AM.png} % Adjust width as needed
		\caption*{\tiny{A Linear Paul Trap from \cite{Leibfried2003-aq}}}
	\end{figure}
	\begin{center}
		$-(\alpha + \beta) = \gamma$\\
		$\alpha'=-\beta'$
	\end{center}
\end{frame}
\begin{frame}{Classical Ion Trapping}
	\begin{itemize}
		\item The problem can be one-dimensionalized:
	\end{itemize}
	\begin{center}
		$\ddot{x}=-\bigfrac{Ze}{m}\bigfrac{\partial V}{\partial x} = -\bigfrac{Ze}{m}\lbrack U\alpha + \tilde{U}\cos(\omega_{rf}t)\alpha'\rbrack x $
	\end{center}
	\begin{itemize}
		\item And transformed into a Mathieu differential equation:
	\end{itemize}
	\begin{center}
		$\bigfrac{d^2x}{d\xi^2} + \lbrack a_x - 2q_x\cos(2\xi)\rbrack x = 0$
	\end{center}
	with
	\begin{center}
		$\xi=\bigfrac{\omega_{rf}t}{2}$,
		\hspace{6pt}
		$a_x=\bigfrac{4ZeU\alpha}{m\omega_{rf}^2}$,
		\hspace{6pt}
		$q_x=\bigfrac{2Ze\tilde{U}\alpha'}{m\omega_{rf}^2}$
	\end{center}
\end{frame}

\begin{frame}{Classical Ion Trapping}
	Mathieu differential equations have solution\\
	\[x(\xi) = Ae^{i\beta_x\xi}\sum_{n=-\infty}^{\infty} C_{2n}e^{i2n\xi} +
		B^{-i\beta_x\xi}\sum_{n=-\infty}^{\infty} C_{2n}e^{-i2n\xi}\]\\
	where  $\beta_x$ and $C_{2n}$ are functions of $a_x$ and $q_x$.\\[8pt]
	Assuming $|a_x|, q_x^2 \ll 1$, $\beta\approx\sqrt{a_x+\frac{q_x^2}{2}}$, and (to first order):
	\[
		x(t)\propto
		\colorbox{cyan}{$\cos(\beta_x\frac{\omega_{rf}}{2}t)$}
		\colorbox{lime}{$\lbrack 1-\frac{q_x}{2}\cos(\omega_{rf}t) \rbrack$}
	\]
\end{frame}

\section{Quantum Ion Trapping}

\begin{frame}{Quantum Ion Trapping}
	Begin with our same potential, but rewrite it:
	\begin{center}
		$V(t)=\bigfrac{m}{2}W(t)\hat{x}^2$
		\hspace{4pt}
		with
		\hspace{4pt}
		$W(t)=\bigfrac{\omega_{rf}^2}{4}\lbrack a_x-2q_x\cos(\omega_{rf}t) \rbrack$
	\end{center}
	This looks familiar! We already know the solutions to
	\begin{center}
		$V(x)=\bigfrac{1}{2}m\omega x^2$
	\end{center}
	Maybe there are some raising and lowering operators...
\end{frame}

\begin{frame}{Quantum Ion Trapping}
	Our Hamiltonian of motion can be written,
	\[\hat{H}^{(m)} = \bigfrac{\hat{p}^2}{2m} + \bigfrac{m}{2}W(t)\hat{x}^2\]
	First, we can write the operators in the Heisenberg picture:
	\begin{center}
		$\dot{\hat{x}} = \bigfrac{1}{i\hbar}\lbrack \hat{x},\hat{H}^{(m)}\rbrack = \bigfrac{\hat{p}}{m}$ \\[4pt]
		$\dot{\hat{p}} = \bigfrac{1}{i\hbar}\lbrack \hat{p},\hat{H}^{(m)}\rbrack = -mW(t)\hat{x}$
	\end{center}
	which combine into a Mattheiu equation!
	\[\ddot{\hat{x}}+W(t)\hat{x} = 0\]
	\[\ddot{\hat{x}}+ \bigfrac{\omega_{rf}^2}{4}\lbrack a_x-2q_x\cos(\omega_{rf}t) \rbrack \hat{x} = 0\]

\end{frame}

\begin{frame}{Quantum Ion Trapping}
	Consider some other $u(t)$ that satisfies the same equation:
	\[\ddot{u}(t)+W(t)\hat{u}(t) = 0\]
	with boundary conditions
	$
		u(0) = 1,
		\hspace{6pt}
		\dot{u}(0) = iv
	$\\[6pt]
	From before, with $A=0$ and $B=1$
	\[u(t) = e^{i\beta_x\omega_{rf}/2} \sum_{n=-\infty}^\infty C_{2n}e^{in\omega_{rf}t} \equiv e^{i\beta_x\omega_{rf}/2} \Phi(t) \]
	Applying the boundary conditions,
	\[
		u(0)=\sum_{n=-\infty}^\infty C_{2n} = 1,
		\hspace{6pt}
		v=\omega_{rf} \sum_{n=-\infty}^\infty C_{2n}(\beta_x/2 + n)
	\]
\end{frame}
\begin{frame}{Quantum Ion Trapping}
	Since $u(t)$ and $\hat{x}$ satisfy the same differential equation, so
	\[\hat{C}(t)=\sqrt{\bigfrac{m}{2\hbar v}}i\lbrack u(t)\dot{\hat{x}} - \dot{u}(t)\hat{x} \rbrack\]
	is proportional to their Wronskian identity, and is constant in time
	\[\hat{C}(t)=\hat{C}(0)=\bigfrac{1}{\sqrt{2m\hbar v}}(mv\hat{x}(0) + i\hat{p}(0)) = \hat{a}\]
	We have recovered exactly the lowering operator!
\end{frame}

\begin{frame}{Quantum Ion Trapping}
	Using this relation, we can write our operators in terms of $\hat{a}$ and $\hat{a}^\dagger$:
	\[\hat{x}(t) = \sqrt{\bigfrac{\hbar}{2mv}} \lbrack\hat{a}u^*(t) + \hat{a}^\dagger u(t)\rbrack \]
	\[\hat{p}(t) = \sqrt{\bigfrac{\hbar m}{2v}} \lbrack\hat{a}\dot{u}^*(t) + \hat{a}^\dagger \dot{u}(t)\rbrack \]
	And just like we're used to, we have a state $\ket{n=0}_v$ such that
	\[ \hat{a}\ket{n=0}_v = 0 \]
	In the Schr\"odinger picture ($\hat{U}(t)=e^{-(i/\hbar) \hat{H}^{(m)}}$):
	\[\hat{C}_S(t)\hat{U}(t)\ket{n=0}_v = \hat{C}_S(t)\ket{n=0,t}=0\]
\end{frame}
\begin{frame}{Quantum Ion Trapping}
	These states form a complete orthonormal basis via the creation operator:
	\[\ket{n,t}=\bigfrac{\lbrack \hat{C}_S^\dagger(t) \rbrack^n}{\sqrt{n!}}\ket{n=0,t}\]
	Excluding a phase factor, the ground state is approximately
	\begin{multline*}
		\chi_0(t)=
		\left(\frac{mv}{\pi\hbar}\right)^{1/4}
		\sqrt{\frac{1+q_x/2}{1+(q_x/2)\cos(\omega_{rf}t)}}\\
		\times
		exp \left( \left\lbrack i\frac{m\omega_{rf}\sin(\omega_{rf}t)}{2\hbar\lbrack 2/q_x + \cos(\omega_{rf}t)]} - \frac{mv}{2\hbar} \right\rbrack x^2\right)
	\end{multline*}
	If $\omega_{rf}=0$, the solution is exactly the harmonic oscillator.
\end{frame}


\section{Quantum Gate Implementation}

\begin{frame}{Rotation Gates}
	We use lasers to change states\\[4pt]
	The total Hamiltonian can be written as a sum of the ion's internal electron state, its motional state, and the interaction with the laser:
	\[\hat{H} = \hat{H}_E + \hat{H}_M + \hat{H}_L\]
	The laser Hamiltonian is defined by  $\Omega$, $\omega$, $k$, and $\phi$
	\[\hat{H}_L=\bigfrac{\hbar\Omega}{2}(\hat{\sigma}^+ + \hat{\sigma}^-)\lbrack e^{i(k\hat{x}-\omega t + \phi)} + e^{-i(k\hat{x}-\omega t + \phi)} \rbrack \]
\end{frame}

\begin{frame}{Rotation Gates}
	But this problem is much easier in the \textit{interaction picture}\\[4pt]
	Let $\hat{H}_0 = \hat{H}_E + \hat{H}_M$ and $\hat{U}_0 = e^{-(i/\hbar)\hat{H}_0t}$, so
	\[\hat{H}_{int} = \hat{U}_0^\dagger\hat{H}_L\hat{U}_0.\]
	After plugging in and much simplification and approximation
	\[
		\hat{H}_{int}(t) = \frac{\hbar\Omega}{2}
		\lbrack
		e^{i\eta(\hat{a}^\dagger e^{ivt} + \hat{a} e^{-ivt})}
		\hat{\sigma}^+
		e^{-i(\textcolor{blue}{(\omega-\omega_0)}t-\phi)}
		+ h.c.
		\rbrack
	\]
	We get resonance when $\omega-\omega_0$ is small and the $\phi$ term dominates
\end{frame}

\begin{frame}{Rotation Gates}
	When the phase term dominates:
	\[
		\hat{H}_{int} = \frac{\hbar\Omega}{2}(\hat{\sigma}^+ e^{i\phi} + \hat{\sigma}^- e^{-i\phi})
		\equiv
		\frac{\hbar\Omega}{2}\hat{\sigma}_\phi
	\]
	And this Hamiltonian describes rotation gates!\\[4pt]
	Let's put this back into the Schr\"odinger equation:
	\[
		\frac{d}{dt}\ket{\psi(t)} = -\frac{i}{\hbar}\Hat{H}\ket{\psi(t)}
		\rightarrow
		\ket{\psi(t)}=e^{-\frac{i\Omega}{2}t\hat{\sigma}_\phi}\ket{\psi(0)}
		\equiv
		\hat{U}(t)\ket{\psi(0)}
	\]
	So we have a unitary operator evolving the system...
\end{frame}

\begin{frame}{Rotation Gates}
	Letting $\Omega t \equiv \theta$:
	\begin{multline*}
		\hat{U}(t) =
		\sum_{k=0}^\infty \frac{1}{k!}(-i\frac{\theta}{2}\hat{\sigma}_\phi)^k =\\
		I\sum_{k=0}^\infty  \frac{(-1)^k}{(2k)!}\left(\frac{\theta}{2}\right)^{2k} -
		i\hat{\sigma}_\phi \sum_{k=0}^\infty \frac{(-1)^k}{(2k+1)!}\left(\frac{\theta}{2}\right)^{2k+1}=\\
		I\cos\left(\frac{\theta}{2}\right) -
		i\hat{\sigma}_\phi\sin\left(\frac{\theta}{2}\right) =
		\begin{pmatrix}
			\cos\frac{\theta}{2}            & -i\sin\frac{\theta}{2}e^{-i\phi} \\
			-i\sin\frac{\theta}{2}e^{i\phi} & \cos\frac{\theta}{2}
		\end{pmatrix}
	\end{multline*}
	We recover the $X$, $Y$, and $Z$ gates with the right choices of $\theta$ and $\phi$!!

\end{frame}

\section{Conclusion}

\begin{frame}{Conclusion}
	\begin{itemize}

		\item Trapped ion systems have incredible physics and we barely scratched the surface:
		      \begin{itemize}
			      \item There are many more gates that have been implemented, including the Hadamard gate and other multi-qubit gates
			      \item This presentation only talked about single ion systems, but many ions can be trapped at once
			      \item There are multiple different kinds of states used as the two state system
		      \end{itemize}
		\item Nonetheless, a large portion of trapped ion dynamics can be described with what we have learned in this class!

	\end{itemize}
\end{frame}

\begin{frame}[allowframebreaks]
	\nocite{*}
	\printbibliography
\end{frame}

\begin{frame}
	\centering
	\Huge{Questions?}
\end{frame}

\end{document}
