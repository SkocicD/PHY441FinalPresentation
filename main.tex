\documentclass{beamer}
\usetheme{metropolis}
\usepackage{graphicx} % Required for inserting images
\usepackage{amsmath} % Required for inserting images
\usepackage{braket}
\usepackage{xcolor}
\usepackage{quantikz}
\usepackage{amssymb} % For \iff

\newcommand{\X}{\begin{bmatrix}0 & 1 \\ 1 & 0\end{bmatrix}}
\newcommand{\Y}{\begin{bmatrix}0 & -i \\ i & 0\end{bmatrix}}
\newcommand{\Z}{\begin{bmatrix}1 & 0 \\ 0 & -1\end{bmatrix}}
\newcommand{\Har}{
    \genfrac{}{}{}{0}{1}{\sqrt{2}}
    \begin{bmatrix}1 & 1 \\ 1 & -1\end{bmatrix}
}
\newcommand{\ketzero}{\ket{0}}
\newcommand{\ketone}{\ket{1}}
\newcommand{\matzero}{\begin{bmatrix}1 \\ 0\end{bmatrix}}
\newcommand{\matone}{\begin{bmatrix}0 \\ 1\end{bmatrix}}

\title{Trapped Ion Quantum Computing}
\author{David Skocic}
\date{December 9 2025}

\begin{document}

\maketitle
\section{Basics of Quantum Computing}
\begin{frame}{Qubits}
	\begin{itemize}
		\item Analog to the classical bit (\textcolor{red}{Qu}\textcolor{blue}{bit} = \textcolor{red}{Qu}antum \textcolor{blue}{bit})
		\item Any two-state quantum system can theoretically be a qubit
		\item We use our familiar Dirac notation to represent states
	\end{itemize}
	\begin{center}
		$\ketzero = \matzero$
		and
		$\ketone = \matone$
	\end{center}

	\begin{itemize}
		\item Qubits can have a superposition unlike a classical bit
	\end{itemize}
	\begin{center}
		$\begin{bmatrix}\alpha \\ \beta\end{bmatrix} = \alpha\ketzero + \beta\ketone$
	\end{center}
\end{frame}
\begin{frame}{Electron Spin}
	\begin{itemize}
		\item Spin is a familiar two-state quantum system
		\item Recall our spin operators for the different directions
	\end{itemize}
	\begin{center}
		$S_x = \genfrac{}{}{}{0}{\hbar}{2}\X$
		\hspace{20pt}
		$S_y = \genfrac{}{}{}{0}{\hbar}{2}\Y$
		\hspace{20pt}
		$S_z = \genfrac{}{}{}{0}{\hbar}{2}\Z$
	\end{center}
\end{frame}

\begin{frame}{Unitary Operators}
	\begin{itemize}
		\item The evolution of a quantum system is described by unitary operators
	\end{itemize}
	\begin{center}
		$\ket{\psi_t} = U(t,t_0)\ket{\psi_0}$
	\end{center}
	\begin{itemize}
		\item Unitary Operators are \textit{mathematically reversible}:
	\end{itemize}
	\begin{center}
		$UU^\dagger=I$
	\end{center}
	\begin{itemize}
		\item You can apply unitary operators to a qubit to alter its state in a desirable manner
	\end{itemize}
\end{frame}

\begin{frame}{Quantum Gates}
	\begin{itemize}
		\item In quantum computing, unitary operators are usually called \textbf{quantum gates}
		\item Many of the basic quantum gates are look very familiar!
	\end{itemize}
	\begin{center}
		$X = \X$
		\hspace{20pt}
		$Y = \Y$
		\hspace{20pt}
		$Z = \Z$
	\end{center}
	\begin{itemize}
		\item There are also less familiar gates such as the \textit{Haramad} Gate:
	\end{itemize}
	\begin{center}
		$H= \Har$
	\end{center}
\end{frame}
\begin{frame}{Quantum Gates}
	\begin{itemize}
		\item The X gate is also called the NOT gate:
	\end{itemize}
	\begin{center}
		$NOT\ketzero=\ketone$ and $NOT\ketone=\ketzero$
		\\[12pt]
		$\X
			\begin{bmatrix}\alpha\\\beta\end{bmatrix} =
			\begin{bmatrix}\beta\\\alpha\end{bmatrix}$
	\end{center}
	\begin{itemize}
		\item The Haramad gate creates a superposition:
	\end{itemize}
	\begin{center}
		$\Har\matzero =
			\genfrac{}{}{}{0}{1}{\sqrt{2}}
			\begin{bmatrix}1\\1\end{bmatrix}=
			\genfrac{}{}{}{0}{1}{\sqrt{2}}
			(\ketzero+\ketone)$
	\end{center}
\end{frame}
\begin{frame}{Quantum Circuits}
	\begin{itemize}
		\item A series of quantum gates constitutes a \textbf{quantum circuit}
		\item It is standard to represent these as if they were on a wire
	\end{itemize}

	\begin{center}
		\begin{quantikz}
			\ketzero & \gate{H} & \gate{Y} & \gate{Z} & \meter{}
		\end{quantikz}
		\\\rotatebox[origin=c]{90}{{$\iff$}}\\
		$MZYH\ketzero=M\Z\Y\Har\matzero=\genfrac{}{}{}{0}{M}{\sqrt{2}}\begin{bmatrix}-i\\-i\end{bmatrix}$
		\\[12pt] Where M represents the act of measurement
	\end{center}
\end{frame}
\section{Trapping an Ion}
\begin{frame}{Classical Ion Trapping}
\end{frame}

\end{document}
